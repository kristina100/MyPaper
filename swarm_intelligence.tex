% Mon Apr 29 2002  15:49:54PM    modified by:  tjn <naughtont@ornl.gov>
%
%  Example using the OLS'02 Style 
%
\documentclass[twocolumn]{article}
\usepackage{ols,epsfig}
\usepackage{graphicx}
\usepackage{algorithm}  
\usepackage{algpseudocode}  
\usepackage{amsmath}  
\usepackage{multirow}
\usepackage{listings}
\usepackage{color}

\definecolor{dkgreen}{rgb}{0,0.6,0}
\definecolor{gray}{rgb}{0.5,0.5,0.5}
\definecolor{mauve}{rgb}{0.58,0,0.82}

\lstset{frame=tb,
	language=Python,
	aboveskip=3mm,
	belowskip=3mm,
	showstringspaces=false,
	columns=flexible,
	basicstyle={\small\ttfamily},
	numbers=none,
	numberstyle=\tiny\color{gray},
	keywordstyle=\color{blue},
	commentstyle=\color{dkgreen},
	stringstyle=\color{mauve},
	breaklines=true,
	breakatwhitespace=true,
	tabsize=3
}

\renewcommand{\algorithmicrequire}{\textbf{Input:}}  % Use Input in the format of Algorithm  
\renewcommand{\algorithmicensure}{\textbf{Output:}}

\begin{document}

\date{}

\title{Optimized swarm Intelligence methods and applications for Large-scale data}

\author{
    Fangqi Nie
}

\twocolumn[
\begin{@twocolumnfalse}
	\maketitle %maketitle必须添加在这里,否则摘要会跳转到下一页
	\begin{abstract}
		In the face of large-scale problems, traditional single-individual or centralized approaches to solving them are no longer sufficient, and researchers have found that groups (societies) exhibit far more complex intelligent behavior than individuals. Therefore, swarm intelligence methods are proposed to bring together swarm intelligence to collaboratively solve large-scale complex problems. First, three evolutionary frameworks for swarm intelligence based on probability distributions are proposed, in which optimization of hierarchical learning swarm intelligence evolution is proposed, and distributed collaborative swarm intelligence evolution optimization is proposed to reduce the time complexity, increase the search diversity, and continuously optimize. Then the application of the corresponding algorithms is proposed, and finally my experience.
		
		
		\textbf{Keywords}
		Ant Colony Algorithm, Particle Swarm Algorithm, Niche, Probability-based  Evolutionary Algorithm, DEGLSO
	\end{abstract}
\end{@twocolumnfalse}
]

\section{Introduction}
Optimization problem is to find the optimal solution among a given
objective function, a given number of solutions, and adjustable
parameters, and to find the most suitable parameter value under certain
constraints. They are widely found in many fields such as signal
processing, image processing, and task assignment. It is an applied
technique based on mathematics for solving various optimization
problems. It is widely used in many of the above-mentioned fields and
has also generated significant economic benefits. The optimization
method has proven to improve system efficiency, reduce energy
consumption, and use resources rationally, and this effect becomes more
obvious as the size of the processing object increases.

It is not difficult to know that in real life, many complex
combinatorial optimization problems constantly arise in various academic
disciplines. In the face of these large optimization problems,
traditional optimization methods transform the multi-objective
optimization problem into a single-objective optimization problem by
certain artificial approach, and then solve the transformed
single-objective optimization problem. The commonly used methods are:
objective weighting method, constraint method and objective programming
method. However, the traditional optimization algorithms can only find
the local optimal solution of the optimization problem, and the result
of the solution strongly depends on the initial value. For example, many
engineering optimization problems often require finding optimal or near
optimal solutions in a complex and large search space. Given the many
characteristics of practical engineering problems, such as complexity,
nonlinearity, constraint and difficulty in modeling, the search for
efficient optimization algorithms has become one of the main research
contents of artificial intelligence.

Inspired by human intelligence, the social nature of biological groups,
or the laws of natural phenomena, swarm intelligence methods have been
proposed: methods that bring together swarm intelligence to
collaboratively solve large-scale complex problems. For example,
multi-optimized single-solution algorithms, mainly including: genetic
algorithms that imitate the evolutionary mechanism of organisms in
nature; Ant Colony Optimization that simulate the collective
path-finding behavior of ants; Particle Swarm Optimization that simulate
the behavior of flocks of birds and schools of fish, and so on. These
algorithms have in common that they are developed by simulating or
revealing certain phenomena and processes in nature or the intelligent
behavior of groups of organisms. In the field of optimization they are
called intelligent optimization algorithms, which are simple, general
and easy to process in parallel. However, traditional intelligent
optimization algorithms suffer from the problem of falling into a local
optimum solution so it is proposed to modify the parameters or use
adaptive modification of the relevant operators to improve the
scalability of the algorithm. While facing the multi-solution
multi-optimization problem, it is proposed that the small habitat
strategy combined with the evolutionary algorithm based on probability
distribution improves the exploration and exploitation ability of the
group intelligent collaborative search through segmentation domination
and hierarchical learning, but the traditional algorithm, may be poor in
performance, so the distributed elite learning strategy is proposed,
which reduces the transmission and time complexity, but still maintains
the group global search and local exploitation capabilities.

This paper is divided into six areas: (I) Introduction (II) Research
background and traditional genetic algorithms (III) Predictability of
swarm intelligence: a framework for the evolution of swarm intelligence
based on probability distribution (V) Scalable swarm intelligence:
distributed collaborative swarm intelligence evolutionary optimization
(VI) swarm intelligence optimization applications (VII) Summary


\section{Research Background and Traditional Genetic
	Algorithms}
Reasons for research on population intelligence methods and applications
for large-scale optimization. First, the national demand, in the "new
generation of artificial intelligence development plan" clearly pointed
out swarm intelligence methods research as one of the five major
development directions. Second, it is a disciplinary frontier, an
important method for solving complex large-scale problems that cannot be
solved by single-individual or centralized methods, with several
branches selected as ESI research frontiers.

\subsection{The Challenge of Large-Scale Problems: Effectiveness and Efficiency}

\subsubsection{Dramatic increase in local optimal}
Global optimum means that a particular decision or choice is the best
among all decisions or choices. The local optimum, on the other hand, is
different from the global optimum in that it is not required to be the
best among all decisions. It means that a particular decision or choice
is the best among a portion of the decisions or choices. It is defined
by a mathematical formula as follows.
\begin{gather}
		f(x_0) = min{f(x)},x_0 \in D \notag\\
		f(x_0) = max{f(x)},x_0 \in D \notag
\end{gather}
However, the problem of "premature convergence" is magnified when the
number of local optimization regions increases sharply. What is
premature convergence? Premature convergence is a phenomenon that cannot
be ignored in genetic algorithms, mainly when all individuals in the
population tend to the same expression and terminate their evolution,
and eventually, the algorithm does not give an excellent solution.

\subsubsection{Exponential expansion of the solution space}

As the size of the group increases, there is an urgent need to improve
search diversity.

\subsubsection{Increased Time Complexity}
Severe reduction in computational efficiency.


\subsection{Traditional Genetic Algorith}
\label{subsec:other}
Back up to another subsection, still within the Discussion section so
that we can continue to inform and amaze.


\textbf{biological terms}

	Genotype: the internal expression of the chromosome of a sex trait.

	Phenotype: the external expression of a chromosomally determined trait
	or, alternatively, the external expression of an individual formed
	according to the genotype.

	Evolution: the gradual adaptation of a population to its living
	environment and the continuous improvement of its quality. The
	evolution of organisms takes place in the form of populations.

	Fitness: a measure of how well a species is adapted to the
	environment in which it lives.

	Selection: Selecting a number of individuals from a population with a
	certain probability. In general, selection is a process of meritocracy
	based on fitness.

	Reproduction: when a cell divides, the genetic material, DNA, is
	transferred to the newly created cell by replication, and the new cell
	inherits the genes of the old cell.

	Crossover: DNA is cut at an identical position on two chromosomes and
	the two strings before and after are crossed and combined to form two
	new chromosomes. Also called genetic recombination or hybridization.
	
	Mutation: some replication errors may (with a small probability) occur
	during replication, and mutations produce new chromosomes that express
	new traits.

	Coding: genetic information in DNA is arranged in a certain pattern on
	a long strand. Genetic coding can be seen as a mapping from expression
	to genotype.

	Decoding (decoding): mapping of genotypes to expressions.

	Individual: an entity whose chromosomes with characteristics.

	Population: the set of individuals, and the number of individuals
	within that set is called the population.


\textbf{Simple process of algorithm implementation}

\begin{enumerate}
	\def\labelenumi{\arabic{enumi}.}
	\item
	Establish the mapping between expression and genotype
	\textless-\textgreater{} digital encoding scheme
	
	Encode as a binary string
	\item
	the first kangaroos are scattered randomly on the mountain range
	\textless-\textgreater{} initialize a population with random numbers
	and the individuals of the population are these digitized codes
	\item
	get the kangaroo location \textless-\textgreater{} decoding process
	\item
	the higher the kangaroo climbs, the more popular it is and the higher
	the fitness \textless-\textgreater{} do a fitness assessment for each
	individual using the fitness function
	
	\textbf{Measure}: the altitude of the kangaroo's location. The
	altitude of the kangaroo can be used directly as their fitness score.
	The fitness function can return the function value directly.
	\item
	Every so often, shoot some kangaroos on the mountain that are at a
	lower elevation, so that the overall number of kangaroos is equal
	\textless-\textgreater{} select on merit with the selection function
	
	\textbf{choice function}
	
		
		Roulette wheel selection: A playback random sampling method in which
		the probability of each individual entering the next generation is
		equal to the ratio of its fitness value to the fitness value of the
		individuals in the whole population.
	
		Random competitive selection: Each time a pair of individuals is
		selected according to the roulette wheel, and then these two
		individuals are allowed to compete, and the one with higher fitness
		is selected, repeatedly, until the selection is full.
	
		Optimal preservation strategy: The individual with the highest
		fitness in the current population does not participate in the
		crossover and variation operations, but uses it to replace the
		individual with the lowest fitness in the current generation
		population after the crossover and variation operations.
	
		Randomized league selection: The individual with the highest fitness
		among several individuals is selected at a time to be inherited into
		the next generation population.
		
		Exclusion selection: the newly generated offspring will replace or
		exclude similar individuals of the old sire, increasing the
		diversity of the population.

	\item
	Individual kangaroo genetic recombination or mutation
	\textless-\textgreater{} generating good offspring
	
	\textbf{Gene recombination/crossover}
	
	The process of gene exchange of binary codes is very similar to the
	process of homologous chromosome association as taught in high school
	biology. A number of codes located at the same position are randomly
	swapped to produce a new individual. In the case of a gene containing
	multiple floating point codes, the floating point number can also be
	used as the base unit to randomly generate a value between the parents
	gene code value as the child gene code value.
	
	\textbf{Crossover} operator for binary-encoded individuals or
	floating-point-encoded individuals.
	

		One-point Crossover: refers to setting only one random crossover
		point in the coding string of an individual, and then exchanging
		part of the chromosomes of two paired individuals with each other at
		that point.

		Two-point Crossover and Multi-point Crossover
		
		\begin{itemize}
			\item
			Two-point crossover: Two crossover points are randomly set in the
			coding strings of individuals, and then some genes are exchanged.
			\item
			Multi-point Crossover : Multiple crossover points are set
		\end{itemize}
	
		Uniform Crossover: Genes at each locus of two paired individuals are
		exchanged with the same crossover probability, resulting in two new
		individuals.
	
		Arithmetic Crossover: Two new individuals are created by the linear
		combination of two individuals. The object of this operation is
		usually an individual represented by a floating-point code.

	
	\textbf{Genetic variation}
	

		Simple Mutation: The mutation operation is performed on the value of
		one or more randomly specified motifs in an individual coding string
		with a mutation probability.

		Uniform Mutation: Replacing the original gene values at each locus
		in an individual coding string with a random number that fits into a
		range of uniform distribution with a small probability. (Especially
		suitable for the initial stage of the algorithm)

		Boundary Mutation: Randomly replaces the original gene value by one
		of the two corresponding boundary gene values in the locus. This is
		particularly suitable for problems where the optimal point is at or
		near the boundary of the feasible solution.

		Non-uniform mutation: A random perturbation of the original gene
		values, and the result of the perturbation is used as the new gene
		value after the mutation. After the mutation operation is performed
		with the same probability for each locus, it is equivalent to a
		slight change of the entire solution vector in the solution space.

		Gaussian approximate variation: the variation operation is performed
		by replacing the original gene value with a random number from a
		normal distribution with sign mean P and variance \$P\^{}2\$.
\end{enumerate}

\textbf{Basic Process}


\begin{figure}
	\centering
	\includegraphics{G:/picture/论文/E遗传.png}
	\caption{Genetic Algorithm}
\end{figure}
This is just a brief look at the most basic genetic algorithms, and what
follows is a description of several commonly used probability
distribution-based frameworks for swarm intelligence evolution.


\section{Population intelligence predictable: a framework for evolution of swarm intelligence based on probability distribution}

\subsection{Single-solution evolutionary algorithm for multi-peaked problems}

Approaching the optimal solution of the problem by iterative evolution
of the population


\begin{algorithm}[H]
	\caption{The Basic Framework of the Evolutionary Algorithm}
	\label{alg:Framwork}
	\begin{algorithmic}[1]
		\Require
		population size: NP, maximum number of iterations: g\_max;
		\Ensure
		global optimal solution
		\State 
		initialize the population and evaluate each individual fitness value;
		\State
		generation = 0;
		\While{generation=g\_max}
		\State
		obtain the offspring individuals based on the parent individuals by
		manipulating the operator, etc;
		\State
		evaluate the fitness values of the offspring individuals;
		\State
		eliminate the poor parent individuals with the child individuals and
		update the global optimal solution Gbest;
		\State
		generation++;
		\EndWhile
	\end{algorithmic}
	\label{code}
\end{algorithm}

No evolutionary algorithm uses a different evolutionary algorithm uses a
different evolutionary mechanism and evolutionary operator, next I will
introduce in detail the ant colony algorithm and particle swarm
optimization algorithm.

\subsection{Description of the basic principles of the ant colony algorithm}


Ant colony algorithm is a classical probabilistic algorithm used to find
the optimal path in 1992, which was proposed by Italian scholars Dorigo
M et al. based on the foraging behavior of ants in nature. When ants
search for food, they will leave unique identification biometric
information -\/-\/-\/-\/-\/- pheromone for communication among their
peers, similar to human language communication. The ants will also
choose their next step according to the pheromone concentration left by
their peers in multiple paths, generally following the path with higher
"pheromone" concentration, and each ant will also leave pheromone on the
path it is walking. As the pheromone concentration increases, ants are
more likely to choose certain paths, thus forming a positive feedback
mechanism.

There are now also monitored experiments on path selection by pheromone
concentration in certain species of ants. For example, the double bridge
experiment designed by Deneubourg and colleagues was completed. It was
obtained that ants use information about changes in their surroundings
to indirectly transmit information as they move, thus adjusting their
own group behavior.

\subparagraph{Important principles of ant colony algorithm implementation}\label{important-principles-of-ant-colony-algorithm-implementation}

\begin{enumerate}
\def\labelenumi{\arabic{enumi}.}
\item
Obstacle blocking: When an obstacle blocks the movement direction of
an ant, the ant will randomly choose another direction (except with
pheromone guidance)
\item
Dispersing pheromones: Each ant emits the most pheromones at the
beginning when it is looking for food, and the pheromones it emits
decrease as it walks more distances.
\item
Range: Ants had a limited observation range
\item
Movement rules: Ants also choose their next steps based on the pheromone concentration left by their companions in multiple paths, generally following the path with higher "pheromone" concentration.
Avoid circling in place.
\item
Foraging rule: If ants find a food source within their sensory range,
they go straight to it instead of taking the path with more
pheromones, and each ant is allowed a small probability of making a
mistake and not going towards the path with the most pheromones to
break the local optimum.
\end{enumerate}

\subparagraph{Traveling salesman problem}\label{traveling-salesman-problem}

A , who travels salesman from his home to a given n cities on business,
wants to find a shortest path to each city and visit each city only
once.

\subparagraph{Ant colony algorithm to solve the TSP problem}


\textbf{Basic Process}

\begin{figure}[H]
	\centering
	\includegraphics[scale=0.85]{G:/picture/论文/遗传E.png}
	\caption{AC}
\end{figure}

See Appendix I for detailed codes.

\subsection{Description of the basic principles of the particle
swarm
algorithm}


Particle swarm optimization (PSO) was proposed by Kennedy and Eberhart
in 1995. The algorithm is a modification of Hepper's model for
simulating a flock of birds (flock of fish) so that the particles can
fly to the solution space and land at the best solution, resulting in a
particle swarm optimization algorithm. The basic core is to use the
sharing of information among the individuals in the flock so that the
motion of the whole flock evolves from disorder to order in the problem
solution space, thus obtaining the optimal solution to the problem.

Imagine a scenario where a flock of birds is foraging and there is a
cornfield in the distance, all birds do not know where the cornfield is
but know how far their current position is from the cornfield. The best
strategy to find the cornfield, and the simplest and most effective one,
is to search the area around the nearest flock of birds, and PSO is an
optimization model inspired by this flock foraging behavior.

In PSO, the solution of each optimization problem is a bird in the
search space, called a "particle", and the optimal solution of the
problem corresponds to the cornfield that the flock is looking for. Each
particle has a position vector (the position of the particle in the
solution space) and a velocity vector (which determines the direction
and speed of the next flight), and can calculate the adaptation value of
its current position according to the objective function, which can be
interpreted as the distance to the "corn field". At each iteration, the
particle learns from its historical position and also from the
"experience" of the best particle in the population to determine how to
adjust and change the flight direction and speed at the next iteration.
After each iteration, the entire population of particles eventually
converges to the optimal solution.

\textbf{Basic framework}

PSO's inertia weighting model

\textbf{Velocity vector}
\[V_{i}^j\to \omega V_{i}^j + c_1r_1^j(PBest_{i}^j - X_{i}^j)+c_2r_2^j(GBest^j-X_i^j)\qquad(1)\]
\[j=1,2,...,n\]
\[w\in[0,1],\ r_1,r_2\in[0,1],\ c_1,c_2=c\]
\[w\ is\ the\ PSO\ inertia\ weight\,\ w=0.9w;\]
\[c_1,c_2\ are\ the\ learning\ factors;\]
\[r_1,r_2\ between\ [0,1]\ random\ probability\ values.\]

\textbf{Position vector}
\[X_i^j = X_i^j+V_i^j\qquad(2)\]

\textbf{Steps}

	\begin{enumerate}
	\def\labelenumii{\arabic{enumii}.}
	\item
	Initialize the population, calculate the individual fitness value,
	select the local optimal position vector Pbest of the individual and
	the global optimal position vector Gbest of the population
	\item
	Set the number of iterations g\_max, and make the current iteration
	g=1
	\item
	Update the individual velocity according to Equation (1)
	\item
	Update the position vector of each individual according to Equation
	(2)
	\item
	Update the local position vector and global position vector: update
	the Pbest of each individual and Gbest of the population
	\item
	When the number of iterations all reach g\_max, output Gbest if it is
	satisfied; otherwise continue to iterate
	\end{enumerate}

\begin{figure}
	\centering
	\includegraphics{G:/picture/论文/粒子群E.png}
	\caption{PSO}
\end{figure}

\textbf{Perception of inertia weights}

The influence of the past motion state of the particle on the current
behavior is similar to the inertia mentioned in physics. If
w\textless\textless1, the previous motion state can rarely affect the
current behavior and the particle's velocity will change quickly; on the
contrary, with larger w, although there will be a large search space, it
is difficult for the particle to change its motion direction and
converge to a better position. A higher w setting promotes global
search, and a lower w setting promotes fast local search.

\textbf{Traveling salesman problem}

See Appendix I for detailed codes.

Traditionally, multi-peaked optimization problems generally refer to
single-solution problems, and such problems generally hope to find the
global optimal solution in the solution space. Due to the large number
of local optimal solutions of the problem, the algorithm is prone to
fall into local optimal solutions and premature convergence, and to
overcome such drawbacks, it is necessary to maintain high search
diversity to help jump out of the local optimal region, and also to save
the number of adaptation value evaluations and improve the efficiency of
the algorithm search.

Scholars have studied and proposed some improvements from three main
perspectives:

\begin{enumerate}
	\def\labelenumi{\arabic{enumi}.}
	\item
	adjusting the parameters of evolutionary algorithms. For example, the
	performance of genetic algorithms is affected by crossover and
	variance probabilities, and particle swarm optimization algorithms are
	affected by inertia weights and learning factors. Also, the choice of
	these control parameters varies for different types of optimization
	problems and at different stages of evolution. Scholars then raise the
	question of how to dynamically and adaptively select appropriate
	control parameters for the algorithm. Currently, there are two types:
	1. evolutionary state-aware adaptive control parameter methods, which
	analyze the adaptation values and distribution characteristics of the
	population, perceive the evolutionary state of the algorithm in the
	evolutionary process, and finally set different parameter adjustment
	strategies according to the characteristics of the evolutionary state
	in which the algorithm is located 2. parameter self-adjustment
	strategies, which adjust the parameters according to the adaptation
	values
	\item
	operator of evolutionary algorithm, Kennedy et al. proposed learning
	from the historical optimal solution lbest of some neighborhood of an
	individual instead of gbest; Cheng et al. proposed a particle swarm
	algorithm with competitive learning for large-scale optimization
	problems. Chen Weineng et al. proposed a dominant learning mechanism
	based on chunking, drawing on the competitive strategy, and introduced
	the PSO algorithm, forming the PSO algorithm based on the chunking
	dominant learning mechanism
	\item
	hybrid evolutionary algorithm, which improves the performance of the
	algorithm by combining multiple evolutionary algorithms.
\end{enumerate}

\subsection{Multi-solution optimization algorithms for multi-peaked problems}

There may be more than one acceptable approximate global optimal
solution in practical applications, then the algorithm needs to find
these solutions as much as possible and provide better decision support,
compared to a single solution, it needs to have a higher search
diversity capability, the current mainstream strategies are: Niche
strategy, adaptive evolution strategy based on probability distribution.
The Niche strategies are described in detail below.

\textbf{Niche} : is a survival environment in a specific environment
where organisms, in their evolutionary process, generally always live
together with their own identical species and reproduce together. A
simple description of the technique is to divide each generation of
individuals into several classes, and select a number of individuals
with greater adaptability in each class as the best representatives of a
class to form a group, and then in the population, and between different
populations, cross mutate to produce a new generation of individual
groups. This Niche genetic algorithm can better maintain the diversity
of solutions, and at the same time has a high global
optimization-seeking ability and convergence speed, which is especially
suitable for optimization problems with complex multi-peaked functions.
The Niche techniques include pre-selection-based Niche habitat
implementation method, exclusion model-based small habitat
implementation method, and shared function-based small habitat
implementation method.

Min Zheng and Junbo Gao had proposed an improved genetic algorithm for
Niche. The algorithm introduces a pre-selection mechanism in the Niche
genetic algorithm based on the elimination similarity mechanism, and
improves the adaptive crossover probability operator and variation
probability operator to dynamically adjust the crossover probability and
variation probability size of individuals according to the size of the
population fitness value.

\textbf{Simple flow chart}

\begin{figure}
	\centering
	\includegraphics{G:/picture/论文/小生境E.png}
	\caption{Swarm Algorithm}
\end{figure}

\textbf{Detailed derivation process}

\begin{enumerate}
	\def\labelenumi{\arabic{enumi}.}
	\item
	Initialize the population to determine the individual coding length:
	GL*VN + VN, M: number of populations, GL: individual coding length, VN
	is the number of independent variables
	\item
	Construct the fitness function:
	
 	\[f(x_1,x_2)=\begin{cases}
	1+0.01*f_s(x_1,x_2) \quad f_s(x_1,x_2)<0 \quad \\
	1 \qquad \qquad \qquad \quad f_s(x_1,x_2)\geq0 \quad(1)
	\end{cases}\]
	\[f(x_1,x_2):Individual\ adaptability\]
	\[f_s(x_1,x_2):The\ value\ of\ the\ function\]
	\[Shubert\ corresponding\ to\ the\]
	\[function\ on\ the\ x_1,x_2\ coordinates.\]
	
	
	\item
	Selection: Pre-selection mechanism, only when the newly generated
	offspring individuals are more adaptive than the parent individuals,
	the generated offspring individuals can replace the parent
	individuals, and then can be inherited to the next generation. Among
	them, the offspring individuals are structurally similar to the parent
	individuals, so the replacement of structurally similar individuals
	can still effectively maintain species diversity.
	\item
	Crossover: If the populations are crossed over with equal probability,
	it does not guarantee the degree of population optimization. In the
	process of evolution, the crossover operator size is adjusted
	according to the fitness value. For example, the crossover probability
	is higher for individuals with small fitness. Later in the
	evolutionary process, the individuals are all close to the average
	fitness level, and a smaller crossover probability at this point is
	beneficial to retain the optimal individuals for the next generation.
	
	\[P_c(k)=\begin{cases}
		0.8 - 0.3(1-\frac{f(k)}{f_{avg}})^2\quad f(k)\leq f_{avg} \quad (2)\\
		0.8 \qquad \qquad \qquad  \qquad \quad f(k)>f_{avg}
	\end{cases}\]
	\[p_c(k):The\ kth\ individual\ crossover\ probability.\]
	\[f_{avg}:Average\ adaptation\ of\ contemporary\]
	\[ populations\]
	\[f(k):The\ kth\ individual\ adaptation.\]
	
	
	\item
	Mutation: In the process of biological variation and evolution, when
	the fitness is significantly lower or higher than the average, the
	probability is that it has mutated. In order to maintain the overall
	optimization of the population, individuals below the average have a
	higher probability of mutation, while those above the average are
	likely to evolve slowly with a small probability.
	
	\[P_m(k)=\begin{cases}
		p_m - 0.05(f(k)-f_{avg})\ f(k)\leq f_{avg}\ \\
		p_m(k)=p_m\qquad \quad f(k)>f_{avg}\quad (3)
	\end{cases}\]
	\[p_m(k):Probability\ of\ the\ kth\ individual\]
	\[variation.\]
	
	
	\item
	Small habitat elimination method: in each generation of the
	population, first compare the distance between two two individuals,
	within a predetermined range L, in comparing the adaptation between
	the two, apply a stronger penalty function to the lower adaptation,
	reduce the adaptation value, after the treatment of individuals in the
	subsequent evolution is more likely to be eliminated. Haiming Distance
	between the i and j:
	
	\[\left\|X_i-X_j\right\|=\sqrt{\sum_{k=1}^{M}(X_{ik}-X_{jk})^2}\quad (4)\]
	\[i=1,2,...,M+M-1\]
	\[j=i+1,...,M+N\]
	\[M:Chromosome\ length\]
	
	
	\item
	Applied to the Shubert function test.
	
	\[minf_s(x_1,x_2)=\left\{
	\sum_{i=1}^5icos\left[(i+1)x_i+i\right]
	\right\}\quad(5)\]
	\[x\left\{\sum_{i=1}^5icos[(i+1)x_2+i]
	\right\}\]
	\[-10\leq x_i \leq 10,\ i=1,2\]
	
	
\end{enumerate}


\section{Group intelligence can be guided: hierarchical learning group intelligence evolutionary optimization}

Hierarchical learning and its advantages.

Each dominant individual can be learned: a large number of role models
are set in the population, maintaining good search diversity.

The higher the level of individuals, the higher the probability of
being learned, the higher level particles concentrate on developing
the solution space, the bottom level particles concentrate on
exploring the solution space, and rely on population self-organization
to achieve equilibrium.

Extension 1: \textbf{Segmented dominance learning}


Based on the competition mechanism, the population is divided into a better set of individuals RG and a worse set of individuals RP.

Segment each worse particle dimension and learn each segment to a better particle in RG
Main value: Improving exploration and exploitation of population
intelligence for collaborative search through hierarchical learning
and segmentation domination.


Extension 2: \textbf{Adaptive Dominance Learning}

Randomly select two individuals, if an individual is dominated by
these two selected individuals, then learn from these two dominated
individuals, otherwise go directly to the next generation, with the
same time complexity as traditional PSO and half the space complexity
of traditional algorithms.

Main value: reduce the space complexity of the method while ensuring
exploration and exploitation capability.


\section{Population Intelligence Scalable: Distributed Collaborative Population Intelligence Evolutionary Optimization}


EDA is a class of evolutionary algorithms that has received a lot of
attention in recent years. The algorithm evaluates the information of
population evolution probability distribution based on the information
of individuals in the current population and gives the sampling of this
probability distribution information to generate children individuals,
and the algorithm framework is as follows:


\begin{algorithm}[H]
	\caption{EDA}
	\label{alg:Framwork}
	\begin{algorithmic}[1]
		\Require
		population size NP, number of individuals M for distribution evaluation, maximum number of iterations g\_max;
		\Ensure
		global optimal solution.
		\State 
		initialize the population and evaluate the fitness value of each individual;
		\State
		generation = 0;
		\While{generation=g\_max}
		\State
		select M individuals from the population, and evaluate the population distribution using these individuals;
		\State
		generate new individuals based on the evaluated distribution model by sampling and evaluating the fitness values of the new individuals;
		\State
		update the global optimal solution by selecting a new population to be generated;
		\State
		generation++;
		\EndWhile
	\end{algorithmic}
	\label{code}
\end{algorithm}


Even though probability-based evolutionary algorithms have advantages in
maintaining search diversity, however, evolutionary algorithms based on
probability distributions have still not been studied and applied in
problems such as multi-solution optimization where search diversity is
highly required. Thus, it has been proposed that by combining the
small-habitat strategy with the probability distribution-based
evolutionary algorithm, a new and effective way to solve complex
multi-peak multi-solution optimization problems is provided.

\textbf{Algorithmic framework}

The idea of probability distribution-based evolutionary algorithm for
multi-solution optimization is to combine the probability distribution
evolution with the small habitat strategy to establish the probability
distribution estimation for the region of high fitness value in the
whole space s search, so as to further improve the search diversity of
the algorithm and to guarantee the solution accuracy in combination with
local search.
	1.population initialization and evaluation of the fitness value of each
	individual.\\
	2.Niche partitioning, using the Niche strategy to partition the
	population into several microhabitats.\\
	3.probability distribution estimation, where the Niche are independent
	of each other and the probability distribution within each Niche is
	independent of each other. Based on the subpopulations corresponding
	to each Niche, the evolutionary distribution within each Niche is
	assessed. So that multiple probability distribution models are
	included within the whole population, corresponding to different
	regions of the solution space.\\
	4.Each Niche merges the parent, and child individuals to obtain a new
	population using a nearest neighbor based elite selection strategy.\\
	5.Local search, in order to improve the solution accuracy, the best
	individuals within each Niche are adaptively searched locally in the
	neighborhood by Gaussian distribution, thus improving the solution
	quality.\\
2-6 Iterate iteratively until the algorithm reaches the termination
condition. The algorithm comes from Probability Distribution Based Evolutionary Computer Optimization.

\begin{algorithm}[H]
	\caption{Niche}
	\label{alg:Framwork}
	\begin{algorithmic}[1]
		\Require
		population size NP, number of groups N, number of local search points K, maximum number of iterations g\_max
		\Ensure
		Whole population
		\State
		generation = 0;
		\While{generation<=g\_max}
		\State
		Dividing the population into N groups, each group contains M = NP/N individuals;
		
		\ForAll{$i \in N$}
		\State
		Evaluate the probability distribution of group i;
		\State $\mu=\frac{\sum_{j=1}^{M}x_j}{M}$,
		$\delta=\sqrt{\frac{\sum_{j=1}^{M}(x_j-\mu)^2}{M-1}}$;
		\State
		\If(rand(0,1)<0.5)
		\State
		Cauchy($\mu,\delta$)
		\Else
		\State
		Gaussian($\mu,\delta$)
		\EndIf
		\ForAll{$j \in M$}
		\State
		Obtain the closest individual to the new individual within the group and replace if better
		\EndFor
		\EndFor
		\State
		Draw the best individuals from each group, put them into the seed set and calculate the probability of performing a local search for each seed
		\ForAll{$ i \in N $}
		\If{$rand(0,1)<P_{r_i}$}
		\ForAll{$j \in K$}
		\State
		$Gaossian(s_i,1.0E-4)$
		If the individual is better, replace;
		\EndFor
		\EndIf
		\EndFor
		\State
		generation++;
		\EndWhile
	\end{algorithmic}
	\label{code}
\end{algorithm}
\textbf{Distributed Collaborative Intelligent Evolutionary Optimization}

Distributed Elite Learning: DEGLSO

Model with central nodes

The role of the central node is to maintain the elite set

The role of an individual is to learn from and evolve the elites
retained by the individual

Communication occurs when and only when individuals fail to discover a
more optimal solution

Key values: low communication overhead, high parallelism granularity,
and still maintain the global exploration and local development
capabilities of the group


\section{ Group intelligence optimization application}

\subsection{Application: Large-scale water supply network optimization}

Objective: Given a water supply pipeline network, select the appropriate
parameters (material, radius) for each section of pipeline and find the
minimum cost design solution that can meet the water supply demand

\textbf{Challenges}

Large scale of pipe network

It is not possible to evaluate the design solution with a precise
expression of the objective function, and a simulation tool is needed
for evaluation (EPANET)

\textbf{Task partitioning}

Determine the initial split based on the flow direction of any initial
solution

Different solutions correspond to different flow directions, posing new
challenges to evolution.

\section{Summary}

The paper took more than a month to write, in fact, strictly speaking
this is a study note rather than an academic paper, all the content is
based on the PowerPoint provided by Ms. Xiaomin Hu, I have almost no
understanding of population intelligence, in which all the proper nouns
appear, I do not understand, so find literature is also based on these
proper nouns as a guide, gradually understand one by one to break.

So why did I choose the direction of genetic algorithm and population
intelligence? Before that I had already had a deep understanding of
computer vision, whether from the official website or from some other
professionals' comments, I was fascinated by the related technologies,
applied in animation or visualized in front of a game fan's eyes. I can
imagine me applying certain technologies to realize AI combined with
audio games or other kinds of games, so that players no longer feel that
they are actually separated from the game by a screen, but can actually
touch and have thoughts. Let the game has a soul. But after all, the
game is virtual. To improve the material life of people does not seem to
help much, of course, this is only on behalf of personal remarks.

Especially the current disaster we are facing - the covid-19, I fully
feel the importance of science and technology, the important role of
science and technology in fighting the epidemic and maintaining the
safety of everyone's life. "The most powerful weapon for mankind to
compete with diseases is science and technology, and mankind cannot
overcome catastrophes and pandemics without scientific development and
technological innovation." For example, I studied group intelligence
this time, multi-solution multi-optimization problem, how to effectively
allocate resources? In particular, some complex system decision-making
tasks in open environments. For example, based on group research to get
vaccines, group development to get software, how to make the efficient
completion of the task in a short period of time, in this information
environment of everything connected, group intelligence is increasingly
useful and affects our lives all the time. He also went from a closed to
open and competitive, now life, future life, optimize the distribution,
optimize the decision, make people's life better and more convenient.
This is the reason why I chose to study computer in the first place,
right?

In returning to this thesis, I roughly learn the genetic algorithm, ant
colony algorithm, particle swarm optimization algorithm and
multi-solution multi-optimization method combining small habitat
strategy with probability-based evolutionary algorithm, and know that
the group intelligence prediction gets to find the most suitable
solution in a certain range of solution space, and also know that the
intelligence of a group is always limited, so I propose to explore the
group intelligence collaborative optimization in distributed environment
We propose to explore collaborative group intelligence optimization in a
distributed environment to improve the efficiency and scalability of
collaboration. However, the traditional master-slave distributed model,
which is a simple population update followed by individual assignment,
adaptation value calculation, etc., results in large communication
volume, long communication waiting time, low granularity of distributed
computation, and poor scalability. But in fact, there are many less
necessary calculations. Scholars then proposed that distributed elite
learning, where communication relationships occur only when individuals
find a more optimal solution, or when individuals fail to evolve over
time, reduces communication overhead and does not have a large impact on
the population's global exploration and local exploitation capabilities.

The whole research process I found fascinating, not enough search power
to enhance, not enough efficiency to find ways to optimize, for the
ultimate better optimization of resources or to be useful in any
scenario that needs to be allocated in life. It's fascinating, but I
also see the challenge in it. Let's end with one of my favorite classic
lines:i went to the woods because i wanted to live deliberately ...i
wanted to live deep and suck out all the marrow of life!\\
to put to rout all that was not life...\\
and not when i came to die, discover that i had not lived...

I will encounter many difficult problems in the process of learning, I
hope I can think more, stand on the shoulders of giants, and learn
without end. Believe in yourself and keep going.
\newpage
\section{References}\label{references}


{[}1{]} Eberhart R C and Kennedy J. A new optimizer using particle swarm
theory. 1995.

{[}2{]} Shi Y and Eberhart R C. A modified particle optimizer. 1998.

{[}3{]} Kennedy J. Particle swarm optimization. 2010.

{[}4{]} Chenbo Zeng Beijing University of Chemical Technology Research
on population intelligence improvement algorithms and applications for
multimodal optimization problems 2020.

{[}5{]} Lu Qing, Liang Changyong, Yang Shanlin Institute of Computer
Network Systems, Hefei University of Technology Adaptive small habitat
genetic algorithm for multimodal function optimization

{[}6{]} Zheng Min, Gao Junbo A small habitat genetic algorithm for
multimodal optimization 2014.

{[}7{]} Gan J, Warwick K. A genetic algorithm with dynamic Nich niche
clustering for multi modal function optimization. Proc. of the IEEE
Conference on Evolutionary Computation. Piscataway, USA. 2001. 215--222

{[}8{]} Zhao Ya-Qin, Zhou Xian-Zhong. A new method of Chinese text
clustering based on small habitat genetic algorithm. Computer
Engineering,2006,32(6):206-208.

{[}9{]} Zhu Xiaorong,Zhang Xinghua. Global optimization of multi-peak
functions based on small habitat genetic algorithm. Journal of Nanjing
University of Technology,2006,28(3):39-43.

{[}10{]} Zhou M, Sun Shudong. Principles and applications of genetic
algorithms. Beijing: National Defense Industry Press, 1999.

{[}11{]} Huang Q., Chen X. X.. The improvement of genetic algorithm for
small habitats. Journal of Beijing University of Technology,
2004,24(8):675-678.

{[}12{]} Zheng S.R.,Lai J.M.,Liu G.L.,Tang G. An improved real number
coding hybrid legacy algorithm. Computer
Applications,2006,26(8):1959-1962.

{[}13{]} Zhang ZZ, Zhang QY. A small-habitat genetic algorithm for exact
optimization. Computer Applications,2005,25(8):1903-1905.



\newpage
\appendix
\section{Appendix}
\subsection{Code I}
\begin{lstlisting}
import random
import numpy as np
import math
	
# Initialize the variable parameters	location= 30 * randn(40, 2)
# Number of ants
num_ant=200 
# Number of cities
num_city=30
# Pheromone impact factor
alpha=1
# Desired impact factor
beta=1 
# Volatility of pheromone
info=0.1 
# Constant
Q=1 
count_iter = 0
iter_max = 500
#dis_new=1000
	
# symmetric matrix, distance between 
def distance_p2p_mat():
	dis_mat=[]
	for i in range(num_city):
		dis_mat_each=[]
		for j in range(num_city):
			dis=math.sqrt(pow(location[i][0]-location[j][0],
			\2)+pow(location[i][1]-location[j][1],2))
			dis_mat_each.append(dis)
		dis_mat.append(dis_mat_each)
	# print(dis_mat)
	return dis_mat
	
# Calculate the distance corresponding to all paths
def cal_newpath(dis_mat,path_new):
	dis_list=[]
	for each in path_new:
		dis=0
		for j in range(num_city-1):
			dis=dis_mat[each[j]][each[j+1]]+dis
		dis=dis_mat[each[num_city-1]][each[0]]+dis
		dis_list.append(dis)
	return dis_list
	
	
# Point-to-point distance matrix
dis_list=distance_p2p_mat()
# Convert to matrix
dis_mat=np.array(dis_list)
# Expectation Matrix
# Diagonal array is added because the divisor cannot be 0
e_mat_init=1.0/(dis_mat+np.diag([10000]*num_city))
diag=np.diag([1.0/10000]*num_city)
# Or make the diagonal element 0
e_mat=e_mat_init-diag
# Initialize the pheromone concentration of each edge, all 1 matrix
pheromone_mat=np.ones((num_city,num_city))
# Initialize each ant path, all starting from city 0
path_mat=np.zeros((num_ant,num_city)).astype(int)
	
	
# while dis_new>400:
while count_iter < iter_max:
	for ant in range(num_ant):
		# All from 0 cities
		visit=0
		# Cities not visited
		unvisit_list=list(range(1,30))
		for j in range(1,num_city):
		# Roulette method to select the next city
		trans_list=[]
		tran_sum=0
		trans=0
		for k in range(len(unvisit_list)):
			trans+=np.power(pheromone_mat[visit]
			\[unvisit_list[k]],alpha)
			\*np.power(e_mat[visit]
			\[unvisit_list[k]],beta)
			trans_list.append(trans)
			tran_sum =trans
		# Generate random numbers
		rand=random.uniform(0,tran_sum)
	
		for t in range(len(trans_list)):
			if(rand <= trans_list[t]):
				visit_next=unvisit_list[t]
				break
			else:
				continue
		# Fill the path matrix
		path_mat[ant,j]=visit_next
		# Update
		unvisit_list.remove(visit_next)
		visit=visit_next
	
	# After filling the path matrix of all ants, count the total distance 	 
	# of each ant
	dis_allant_list=cal_newpath(dis_mat,path_mat)
	
	# Update shortest distance and shortest path for each iteration       
	if count_iter == 0:
		dis_new=min(dis_allant_list)
		path_new=path_mat[dis_allant_list.index(dis_new)].copy()      
	else:
		if min(dis_allant_list) < dis_new:
			dis_new=min(dis_allant_list)
			path_new=
			path_mat[dis_allant_list.index(dis_new)].copy() 
	
	# Update the pheromone matrix
	pheromone_change=np.zeros((num_city,num_city))
	for i in range(num_ant):
		for j in range(num_city-1):
			pheromone_change[path_mat[i,j]]
			[path_mat[i,j+1]] += Q/dis_mat[path_mat[i,j]][path_mat[i,j+1]]
		pheromone_change[path_mat[i,
		num_city-1]][path_mat[i,0]] += Q/dis_mat[path_mat[i,
		num_city-1]][path_mat[i,0]]
	pheromone_mat=
	(1-info)*pheromone_mat+pheromone_change
	# Iteration count + 1, go to next iteration
	count_iter += 1
	
print("Shortest distance:",dis_new)
print("Shortest distance:",path_new)
	
\end{lstlisting}

\section{Appendix}
\subsection{Code II}
\begin{lstlisting}
import numpy as np
import matplotlib.pyplot as plt
import math

def getweight():
	# inertia weight
	w = 1
	return w

def getlearningrate():
	# The learning factor, also known as the acceleration constant
	c = (0.4961,1.4961)
	return c

def getmaxgen():
	# Maximum number of iterations
	g_max = 300
	return g_max

def getsizepop():
	# population size
	sizepop = 50
	return sizepop

def getrangepop():
	# Range limits for the particle's position, same limits for x and y    	   # directions
	
	rangepop = (-2*math.pi , 2*math.pi)
	#rangepop = (-2,2)
	return rangepop

def getrangespeed():
	# The speed range limit of the particle
	rangespeed = (-0.5,0.5)
	return rangespeed

def func(x):
	# x input particle position
	# y particle fitness value
	if (x[0]==0)&(x[1]==0):
		y = 	np.exp((np.cos(2*np.pi*x[0])
		\+np.cos(2*np.pi*x[1]))/2)-2.71289
	else:
		y = np.sin(np.sqrt(x[0]**2+x[1]**2))
		/np.sqrt(x[0]**2+x[1]**2)
		\+np.exp((np.cos(2*np.pi*x[0])
		\+np.cos(2*np.pi*x[1]))/2)-2.71289
	return y

# Initialize the particle position, velocity, and adaptation values
def initpopvfit(sizepop):
	pop = np.zeros((sizepop,2))
	v = np.zeros((sizepop,2))
	fitness = np.zeros(sizepop)

	for i in range(sizepop):
		pop[i] = [(np.random.rand()-0.5)*rangepop[0]*2,
		\(np.random.rand()-0.5)*rangepop[1]*2]
		v[i] = [(np.random.rand()-0.5)*rangepop[0]*2,
		\(np.random.rand()-0.5)*rangepop[1]*2]
		fitness[i] = func(pop[i])
	return pop,v,fitness

def getinitbest(fitness,pop):
	# The population optimal particle positions and their fitness values
	gbestpop,gbestfitness = pop[fitness.argmax()].copy(),fitness.max()
	# individual optimal particle position and its fitness value, use copy() 	
	# so that changes to pop do not affect pbestpop, pbestfitness similar
	pbestpop,pbestfitness = pop.copy(),fitness.copy()

	return gbestpop,gbestfitness,pbestpop,pbestfitness  

# weights
w = getweight()
# learning factor
c = getlearningrate()
# maximum number of iterations
g_max = getmaxgen()
# population size
sizepop = getsizepop()
# Location range limits
rangepop = getrangepop()
# Speed range limits
rangespeed = getrangespeed()
# Initialize particle position, velocity, adaptation values
pop,v,fitness = initpopvfit(sizepop)
# Optimal particle position and fitness values
gbestpop,gbestfitness,pbestpop,pbestfitness = getinitbest(fitness,pop)
# Result
result = np.zeros(g_max)
for i in range(g_max):
	t=0.5
	# update speed
	for j in range(sizepop):
		v[j] += c[0]*np.random.rand()*(pbestpop[j]-pop[j])
		\+c[1]*np.random.rand()*(gbestpop-pop[j])
		v[v<rangespeed[0]] = rangespeed[0]
		v[v>rangespeed[1]] = rangespeed[1]

	# update location
	for j in range(sizepop):
		#pop[j] += 0.5*v[j]
		pop[j] = t*(0.5*v[j])+(1-t)*pop[j]
		pop[pop<rangepop[0]] = rangepop[0]
		pop[pop>rangepop[1]] = rangepop[1]

	#update fitness
	for j in range(sizepop):
		fitness[j] = func(pop[j])

	for j in range(sizepop):
		if fitness[j] > pbestfitness[j]:
			pbestfitness[j] = fitness[j]
			pbestpop[j] = pop[j].copy()

	if pbestfitness.max() > gbestfitness :
		gbestfitness = pbestfitness.max()
		gbestpop = pop[pbestfitness.argmax()].copy()

	result[i] = gbestfitness

plt.plot(result)
plt.show()
\end{lstlisting}
\end{document}










